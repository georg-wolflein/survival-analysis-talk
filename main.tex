\documentclass{beamer}

\usepackage[skins,theorems]{tcolorbox}
\usepackage{amsmath}
\usepackage{xcolor}
\usepackage{tikz}


\title[Survival analysis]{An introduction to survival analysis}
\author{Georg W\"olflein}
\institute[www.st-andrews.ac.uk]{School of Computer Science, University of St Andrews}
\titlegraphic{\includegraphics[width=0.1\paperwidth]{./figs/crest.pdf}}


\usetheme{Berlin}

\usetikzlibrary{decorations.pathreplacing,calc}
\newcommand{\tikzmark}[1]{\tikz[overlay,remember picture] \node (#1) {};}

\newcommand{\prob}[1]{\ensuremath{\Pr{\left(#1\right)}}}

%%%%%%%%%%%%%%%%%%%%%%%%%%%%%%%%%%%%%

\definecolor{stablue}{HTML}{0052cc}
% \definecolor{stared}{HTML}{ff4d4d}
\definecolor{stared}{HTML}{e60d1b}
\newcommand{\empha}[1]{\textbf{\textcolor{stablue}{#1}}}
\setbeamercolor{frametitle}{fg=stablue,bg=white}
\setbeamercolor{title}{fg=stablue,bg=white}
\setbeamercolor{local structure}{fg=stablue}
\setbeamercolor{section in toc}{fg=stablue,bg=white}
\setbeamercolor{subsection in toc}{fg=stared,bg=white}
\setbeamercolor{item projected}{fg=stablue,bg=white}
\setbeamertemplate{itemize item}{\color{stablue}$\bullet$}
\setbeamertemplate{itemize subitem}{\color{stablue}\scriptsize{$\bullet$}}
\setbeamercolor{block title}{bg=stared!90,fg=white}

\setbeamercolor{section in head/foot}{bg=stablue,fg=white}
\setbeamercolor{subsection in head/foot}{bg=white,fg=black}

\setbeamercolor{footercl}{fg=white,bg=stablue}
\setbeamerfont{stafont}{size = \large}
\setbeamerfont{footerfont}{size = \tiny}
\setbeamertemplate{footline}
{
  \leavevmode%
  \hbox{%
  \fontsize{13}{13}\fontfamily{ppl}\selectfont
  \begin{beamercolorbox}[wd=.5\paperwidth,ht=2.25ex,dp=1ex,left]{footercl}%
    \usebeamerfont{author in head/foot}\hspace*{1ex}\insertshortinstitute
   \end{beamercolorbox}%
   \begin{beamercolorbox}[wd=.25\paperwidth,ht=2.25ex,dp=1ex,right]{footercl}%
    \parbox{.25\paperwidth}{\fontfamily{cmss}\selectfont{\hfill\hfill \usebeamerfont{footerfont}\insertshorttitle~\insertframenumber{}}}
  \end{beamercolorbox}%
  \begin{beamercolorbox}[wd=.25\paperwidth,ht=2.25ex,dp=1ex,left]{footercl}%
    \parbox{.25\paperwidth}{\hfill\includegraphics[height=1cm]{./figs/stalogo.png}}% original: 2ex
  \end{beamercolorbox}}%
  \vskip0pt%
}
\makeatother


\begin{document}

\begin{frame}
    \titlepage
\end{frame}

\begin{frame}{Contents}
    \tableofcontents
\end{frame}

\section{Time-to-event data}

\begin{frame}{What is time-to-event (TTE) data?}
    We can measure \textbf{time} in:
    \begin{itemize}
        \item years
        \item months
        \item seconds
    \end{itemize}
    \pause
    The \textbf{event} could be:
    \begin{itemize}
        \item death from disease \tikzmark{topbrace}
        \item product failure
        \item losing a customer \tikzmark{bottombrace}
    \end{itemize}
    \pause
    \begin{tikzpicture}[overlay, remember picture]
    \draw [decoration={brace,amplitude=0.5em},decorate,black]
        let \p1=(topbrace), \p2=(bottombrace) in
        ({max(\x1,\x2)}, {\y1+0.8em}) -- node[right=0.6em] {must be a binary variable} ({max(\x1,\x2)}, {\y2});
    \end{tikzpicture}
    \pause
    TTE data consists of $(time, \overbrace{event}^{\text{yes/no}})$ tuples.
\end{frame}

\begin{frame}{Time-to-event (TTE) data}
    TTE analysis is also known as:
    \begin{itemize}
        \item survival analysis
        \item failure time analysis
        \item reliability theory (engineering)
        \item duration modelling (economics)
        \item event history analysis (sociology)
    \end{itemize}
    \pause
    Use cases for TTE analysis:
    \begin{itemize}
        \item TODO
    \end{itemize}
\end{frame}

\begin{frame}{Example: Covid-19 treatment trial}
    \only<+>{
        A randomised controlled trial ($n=4$) was conducted to assess the efficacy of drug ABC in treating Covid-19.
        This is what happened to the patients:\bigskip
    }
    \begin{tabular}{c|c|cl}
        patient & received ABC? & outcome \\
        \hline
        1 & yes & died from Covid-19 on day 14 \\
        2 & no & dropped out of the study after day 3 \\
        3 & yes & died by a lightning stroke on day 5 \\
        4 & no & survived the study (90 days) \\
    \end{tabular}\bigskip
    \only<2->{
    The \textbf{time} is the number of days since testing positive for Covid-19.

    The \textbf{event} is whether the patient died due to Covid-19.
    }
    \pause
    \begin{block}{Time-to-event data}
        \centering
        \begin{tabular}{c|c|c}
            patient & time & event \\
            \hline
            1 & 14 & yes \\
            2 & \only<3>{?}\only<4>{$[0,3]$} & \only<3>{?}\only<4>{no} \\
            3 & \only<3>{?}\only<4>{$[0,5)$} & \only<3>{?}\only<4>{no} \\
            4 & \only<3>{?}\only<4>{$[0,30]$} & no \\
        \end{tabular}
    \end{block}
    

\end{frame}

\begin{frame}{Censoring}
    We just saw an example of \textbf{right-censored} data.

\end{frame}

\begin{frame}{}
    \begin{block}{Survival function}
        $$S(t) = \prob{T > t}$$
    \end{block}
    Supposing an individual survived until time $t$, the \textbf{hazard function} expresses the probability of surviving an additional time $dt$.
    \begin{block}{Hazard function}
        $$
            \lambda(t)
            = \lim_{dt\to 0}
            \frac{\prob{t \leq T \leq dt | T \geq t}}{dt}
            = \lim_{dt\to 0}
            \frac{\prob{t \leq T \leq dt}}{dt \cdot S(t)}
        $$
    \end{block}
\end{frame}

\end{document}